\documentclass{article}
\usepackage[utf8]{inputenc}

\usepackage{amssymb, amsmath, amsthm}

\newtheorem{theorem}{Theorem}[section]
\newtheorem{corollary}{Corollary}[theorem]
\newtheorem{lemma}[theorem]{Lemma}
% Remarks are unordered
% To unorder something, just add a '*' to before the {} and remove [section]
% Vice versa to order it
\theoremstyle{definition}
\newtheorem{example}{Example}[section]
\newtheorem{definition}{Definition}[section]
\theoremstyle{remark}
\newtheorem*{remark}{Remark}

% Useful macros
% Updated on Feb 1. 2016 - 
% Removed all my old analysis macros, adding algebra ones

% Turns the square black at the end of the proof. 
% No reason, I just like it better.
\renewcommand\qedsymbol{$\blacksquare$}

% Set of Naturals/Ints/Rats/etc
\newcommand{\N}{\mathbb{N}}
\newcommand{\Z}{\mathbb{Z}}
\newcommand{\Q}{\mathbb{Q}}
\newcommand{\R}{\mathbb{R}}
\newcommand{\C}{\mathbb{C}}

% LaTeX's \overline sucks \overbar{x} fits a bit better
\newcommand{\overbar}[1]{\mkern 1.5mu\overline{\mkern-1.5mu#1\mkern-1.5mu}\mkern 1.5mu}

% Note: The use of the above macros is intentionally avoided
%       below so that all of these macros can be used without 
%       relying on the others
% Easy way to write Z/nZ: \modclass{n}
\newcommand{\modclass}[1]{\mathbb{Z}/#1\mathbb{Z}}
% Easy way to write (Z/nZ)^x \modunits{n}
\newcommand{\modunits}[1]{(\mathbb{Z}/#1\mathbb{Z})^{\times}}

% End of macros

\title{Useful Headers}
\author{Mathew Cha}

\begin{document}

\maketitle

$\N \quad \Z \quad \Q \quad \R \quad \C \quad \overbar{n} \quad \modclass{n} \quad \modunits{n}$\\
\begin{theorem}
Every non-constant single-variable polynomial with complex coefficients has at least one complex root. 
\end{theorem}

\begin{proof}
Outside the reach of this course.
\end{proof}

\begin{corollary}
This is a corollary
\end{corollary}

\begin{lemma}
Likewise, a lemma
\end{lemma}

\begin{definition}
A \textbf{group} is a set and binary operation with the following properties:
\end{definition}

\begin{example}
$\Z$ under addition is a group
\end{example}

\begin{remark}
Remarks are useful
\end{remark}

\end{document}
